\documentclass[12pt, a4paper]{article}
\usepackage{fontspec} % loaded by polyglossia, but included here for transparency 
\usepackage{polyglossia}
\usepackage{microtype}
\usepackage{hyperref}
\usepackage{xifthen}
\hypersetup{
	colorlinks=true, linkcolor = {black}, urlcolor = {blue}
}
\setlength{\parskip}{.5em}
\setmainlanguage{russian} 
\setotherlanguage{english}

% XeLaTeX can use any font installed in your system fonts folder
% Linux Libertine in the next line can be replaced with any 
% OpenType or TrueType font that supports the Cyrillic script.

\newcounter{problem}
\newcommand{\problem}[1]{\refstepcounter{problem}{\bf Задача \theproblem.} \ifthenelse{\isempty{#1}{}}{}{\label{#1}}}

\newfontfamily\russianfont{Times New Roman}
\newfontfamily\englishfont{Times New Roman}
\setmonofont{Courier New}
\newfontfamily{\cyrillicfonttt}{Courier New}

\title{Задания по курсу Python\\Задание 2}
\author{Д.В. Иртегов}
\date{\today}

\begin{document}
\pagestyle{empty}
\maketitle
{\small Задачи необходимо сдать до 24 марта.  Решения необходимо сдавать путем отправки pull request в каталог problems-2 репозитория \\
\url{https://github.com/dmitry-irtegov/NSUPython2018}. 
\\Датой сдачи задания считается дата отправки первого pull request.  Если запрос не принят из-за моих замечаний, у вас есть неделя на их исправление. 

Если запрос принят, задание считается засчитанным.  Если запрос не принят, в комментарии вы можете узнать, почему.

В одном запросе следует отправлять не более одного решения. Если решение состоит из нескольких файлов, в запрос должны быть включены они все.  Все запросы одного студента должны отправляться в каталог с именем, соответствующим его учетной записи.  Например, для задачи 3 из группы задач 2, сдаваемой студентом v-pupkin, рекомендуемое имя файла \verb|problems-2/v-pupkin/task3.py|. }

\problem{} Напишите скрипт, который требует ввода числа из stdin (стандартного потока ввода). Если введенная строка не является числом, скрипт должен требовать
повторить ввод.  Скрипт завершается, когда введено число, или когда введен конец файла (CTRL-Z в Windows, CTRL-D в начале строки в Unix).

\problem{vector} Реализуйте класс Vector, соответствующий N-мерному вектору линейной алгебры.  У этого класса должны быть определены все естественные для вектора операции -- сложение, вычитание, умножение на константу, скалярное произведение и сравнение на равенство, -- а также операции вычисления длины, получение элемента по индексу и строковое представление. Во всех операциях можно считать, что все передаваемые аргументы корректны.

\problem{} Создайте документацию для класса Vector и его методов из задания \ref{vector}. В Python нет закрепленного общего стиля для документации, но можно выбрать 
какой-нибудь популярный и использовать его. Например, можно взять стиль для SciPy \url{https://github.com/numpy/numpy/blob/master/doc/HOWTO_DOCUMENT.rst.txt}
(для начала его можно не читать полностью, а только посмотреть примеры для классов и функций).

\problem{} {\bf Опхряб1:} Напишите программу, которая определяет кодировку текста.  На вход программы подается файл в неизвестной кодировке. Постройте диаграммы частот встречаемости всех не-ASCII символов (байтов, значение которого превосходит 127) в файле.  Программа должна вывести частоты встречаемости каждого байта, в порядке убывания встречаемости (самый часто встречающийся байт идет первым).  Встречаемость определяется по формуле $p[b]*100/N$, где $p[b]$ - количество раз, которые байт встретился в тексте, а $N$ - количество байт с кодами > 127 в тексте.
\\
{\it Замечание:} Ээлемент байтового массива имеет тип int.
\\{\it Замечание 2:} Для проверки правдоподобия значений, вы можете попытаться сконвертировать байиы в символ русского языка, используя код:
\\ \verb|char=bytes([b]).decode('koi8_r')|
\\ где \verb|b| -- значение байта.
\\ Разумеется, это возможно только если вы знаете кодировку файла.
\\ Таблица встречаемости символов русского языка доступна по адресу 
\\ \url{https://www.dpva.ru/Guide/GuideUnitsAlphabets/Alphabets/FrequencyRuLetters/}
\\{\it Замечание 3:} Для генерации тестовых данных, в Linux можно использовать утилиту iconv, а в Windows -- сохранение из редактора Far Manager с переключением кодировки.

\end{document}
