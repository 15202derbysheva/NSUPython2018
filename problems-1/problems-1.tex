\documentclass[12pt, a4paper]{article}
\usepackage{fontspec} % loaded by polyglossia, but included here for transparency 
\usepackage{polyglossia}
\usepackage{microtype}
\usepackage{hyperref}
\usepackage{xifthen}
\hypersetup{
	colorlinks=true, linkcolor = {black}, urlcolor = {blue}
}
\setlength{\parskip}{.5em}
\setmainlanguage{russian} 
\setotherlanguage{english}

% XeLaTeX can use any font installed in your system fonts folder
% Linux Libertine in the next line can be replaced with any 
% OpenType or TrueType font that supports the Cyrillic script.

\newcounter{problem}
\newcommand{\problem}[1]{\refstepcounter{problem}{\bf Задача \theproblem.} \ifthenelse{\isempty{#1}{}}{}{\label{#1}}}

\newfontfamily\russianfont{Times New Roman}
\newfontfamily\englishfont{Times New Roman}
\setmonofont{Courier New}
\newfontfamily{\cyrillicfonttt}{Courier New}

\title{Задания по курсу Python\\Задание 1}
\author{Д.В. Иртегов}
\date{\today}

\begin{document}
\pagestyle{empty}
\maketitle
{\small Задачи необходимо сдать до 3 марта.  Решения необходимо сдавать путем отправки pull request в каталог problems-1 репозитория \\
\url{https://github.com/dmitry-irtegov/NSUPython2018}. 
\\Датой сдачи задания считается дата отправки первого pull request.  Если запрос не принят из-за моих замечаний, у вас есть неделя на их исправление. 

Если запрос принят, задание считается засчитанным.  Если запрос не принят, в комментарии вы можете узнать, почему.

В одном запросе следует отправлять не более одного решения. Если решение состоит из нескольких файлов, в запрос должны быть включены они все.  Все запросы одного студента должны отправляться в каталог с именем, соответствующим его учетной записи.  Например, для задачи 3 из группы задач 1, сдаваемой студентом v-pupkin, рекомендуемое имя файла \verb|problems-1/v-pupkin/task3.py|. }

\problem{} Напишите функцию, разлагающую данное число на простые множители. Результатом
работы программы должен быть список, в котором каждому простому множителю $p$ и
его степени $k$ соответствует пара $(p, k)$ (вложенный список). Например, число 12 должно
быть разложено так: 
\\ \verb|[[2, 2], [3, 1]]|

\problem{isprime} Напишите функцию, которая возвращает True, если данное целое число является простым, и False, если составным. Эта задача не считается (0~баллов), но полезна при реализации других задач.

\problem{} Реализуйте поиск простых чисел в диапазоне от 1 до $N$ решетом Эратосфена.  Выполните реализации решета в виде списка (list), в виде множества (set) и
в виде битового массива (\href{https://pypi.python.org/pypi/bitarray}{bitarray}). Сравните их по производительности и памяти при $N=100 000 000$. 
\\ Для отладки можно использовать функцию, реализованную в задаче \ref{isprime}.
\pagebreak

\problem{} Напишите программу, которая выписывает список файлов в данной директории и
сортирует их в соответствии с их размером. Программа должна получать путь к директории
в качестве аргумента командной строки и печатать на экран имена всех файлов в ней и
их размеры, причем первыми должны идти файлы с наибольшими размерами, а в случае
одинакового размер файлы сортируются по алфавиту.
\\ \emph{Указание.} Изучите функции \texttt{listdir} и \texttt{stat} из модуля \texttt{os} 
и функции\\ \texttt{isfile} и \texttt{join} из модуля \texttt{os.path}.

\problem{} Напишите списковое выражение, генерирующее все простые числа не больше заданного числа.
Можно использовать функцию, реализованную в задаче \ref{isprime}.

\end{document}
